% -*- coding: utf-8 -*-

\documentclass[12pt,plain,most]{randexam}
\geometry{b5paper,margin=2cm}

\newcommand*{\myversion}{2024A}
\newcommand*{\mydate}{\the\year-\mylpad\month-\mylpad\day}
\newcommand*{\mylpad}[1]{\ifnum#1<10 0\the#1\else\the#1\fi}

\cfoot{\small\thepage}

\setlength{\parindent}{0pt}
\setlength{\parskip}{7pt plus 1pt minus 1pt}

\usepackage{arevtext}
\usepackage{iftex}
\usepackage{ninecolors}
\usepackage{hyperref}
\hypersetup{
  colorlinks=true,
  urlcolor=blue3,
  linkcolor=blue3,
}

\renewcommand\familydefault{\sfdefault}

\renewcommand{\baselinestretch}{1}
\renewcommand{\arraystretch}{1.3}

\usepackage{tabularx}

\newcommand{\fillbox}[1]{\ulinefill{#1}\underline{#1}\ulinefill{#1}}

\usepackage{fancyvrb}

\DefineVerbatimEnvironment{code}{Verbatim}{%
  formatcom=\color{blue!50!red}%
}

\usepackage{codehigh}

\NewDocumentCommand\mypkg{m}{\textcolor{blue3}{\mbox{#1}}}
\NewDocumentCommand\myopt{m}{\textcolor{yellow3}{\mbox{#1}}}
\NewDocumentCommand\mycmd{m}{\textcolor{green3}{\ttfamily\fakeverb{#1}}}
\NewDocumentCommand\myenv{m}{\textcolor{green3}{\ttfamily#1}}
\NewDocumentCommand\myfile{m}{\textcolor{purple3}{\mbox{#1}}}

\begin{document}

{%
  \renewcommand{\arraystretch}{2}%
  \noindent\Large
  \begin{tabularx}{\linewidth}{|X|}
    \hline
    Title: \fillbox{\color{blue3}The randexam class for LaTeX}\\
    Author: \fillbox{Jianrui Lyu (tolvjr@163.com)}\\
    Version: \fillbox{\myversion{} (\mydate)}\\
    \hline
  \end{tabularx}%
}

\tableofcontents

\section{Introduction}

Document class \mypkg{randexam} is an exam class for LaTeX.
With this class you could easily make an exam (especially a math exam) and its randomized variants.

The latest release of this package can be downloaded from here: \newline
\textcolor{blue}{\href{https://ctan.org/pkg/randexam}{\ttfamily https://ctan.org/pkg/randexam}}.

\section{Structure}

\subsection{Basic structure}

The following is the basic structure of a \mypkg{randexam} document:
\begin{code}
\documentclass{randexam}
% document preamble
\begin{document}
% document body
\end{document}
\end{code}
In document preamble you could make some settings for the exam.
In document body you write the contents of the exam.

\subsection{Exam settings}

In document preamble you could set the random seed,
which is used when you add class option \myopt{random}:
\begin{code}
\setexam{
  seed = 19061116, % random seed
}
\end{code}

\subsection{Exam body}

The problems in an exam could be separated into several groups:

\begin{code}
\documentclass{randexam}
\begin{document}
......
\makehead % make exam title
\gradetable
......
\makepart{Fill in the blanks.}{3 points for each.}
......
\makepart{Select one answer.}{3 points for each.}
......
\makepart{Work out math calculations.}{8 points for each.}
......
\makepart{Work out math proofs.}{8 points for each.}
......
\makedata{Some data may be used in the exam}
......
\end{document}
\end{code}

\subsection{Exam title}

Before calling \mycmd{\makehead} command,
you need to provide some informations of the exam:

\begin{code}
\renewcommand{\examtitle}{Math 1906 Final Exam}
\renewcommand{\examdate}{2018-06-28}
\renewcommand{\examvariant}{A}
\makehead % make exam title
\end{code}

\subsection{True-or-false problems}

\begin{code}
\makepart{True-or-false problems}{3 points for each.}

\begin{problem}
The first true-or-false problem. \tickout{T}
\end{problem}

\begin{problem}
The second true-or-false problem. \tickout{F}
\end{problem}
\end{code}

\begin{problem}
The first true-or-false problem. \tickout{T}
\end{problem}

\begin{problem}
The second true-or-false problem. \tickout{F}
\end{problem}

With \mycmd{\tickout{T}} and \mycmd{\tickout{F}}, you get \textcolor{blue}{\textsf{T}}
and \textcolor{blue}{\textsf{F}}; with \mycmd{\tickout{t}} and \mycmd{\tickout{f}},
you get \textcolor{blue}{$\checkmark$} and \textcolor{blue}{\large$\times$}.

You must put answers inside \mycmd{\tickout} command,
so that \mypkg{randexam} could hide them in generating empty exam.

\subsection{Fill-in-the-blank problems}

\begin{code}
\makepart{Fill in the blanks.}{3 points for each.}

\begin{problem}
The first fill-in-the-blank problem \fillout{answer}.
\end{problem}

\begin{problem}
The second fill-in-the-blank problem \fillout{answer}.
\end{problem}
\end{code}

\begin{problem}
The first fill-in-the-blank problem \fillout{answer}.
\end{problem}

\begin{problem}
The second fill-in-the-blank problem \fillout{answer}.
\end{problem}

With \mycmd{\fillout} command, the underline will fill the whole line;
with \mycmd{\fillin} command, the underline will be minimal.

You must put answers inside \mycmd{\fillout} or \mycmd{\fillin} command,
so that \mypkg{randexam} could hide them in generating empty exam.

\subsection{Multiple-choice problems}

\begin{code}
\makepart{Select one answer.}{3 points for each.}

\begin{problem}
The first multiple-choice problems \pickout{A}.
\begin{abcd}
  \item First
  \item Second
  \item Third
  \item Fourth
\end{abcd}
\end{problem}

\begin{problem}
The second multiple-choice problems \pickout{C}.
\begin{abcd}
  \item First choice
  \item Second choice
  \item Third choice
  \item Fourth choice
\end{abcd}
\end{problem}
\end{code}

\begin{problem}
The first multiple-choice problems \pickout{A}.
\begin{abcd}
  \item First
  \item Second
  \item Third
  \item Fourth
\end{abcd}
\end{problem}

\begin{problem}
The second multiple-choice problems \pickout{C}.
\begin{abcd}
  \item First choice
  \item Second choice
  \item Third choice
  \item Fourth choice
\end{abcd}
\end{problem}

With \mycmd{\pickout} command, the answer will be printed on the right edge of the line;
with \mycmd{\pickin} command, the answer will be printed on current position.

You must put answers inside \mycmd{\pickout} or \mycmd{\pickin} command,
so that \mypkg{randexam} could hide them in generating empty exam.

The four choices of multiple-choice problems could be typeset with \myenv{abcd} environment.
And \myenv{abcd} environment will put them in one, two, or four rows
according to the lengths of the choices.

\subsection{Answer tables}

Before true-or-false, fill-in-the-blank, or multiple-choice problems,
you may use \mycmd{\answertable} to generate an empty answer table:

\begin{code}
\answertable[3em]{6}{3}
\end{code}

\answertable[3em]{6}{3}

The meanings of the three arguments of \verb!\answertable! commands are as follows:
\begin{itemize}
  \item The first one means the strut height of the cells; its default value is \verb!1em!.
  \item The second one means the total number of problems in this exam group.
  \item The third one means the number of problems in each row.
\end{itemize}

\subsection{Calculation problems}

\begin{code}
\makepart{Work out math calculations.}{8 points for each.}

\begin{problem}
The first math calculation problem.
\end{problem}
\begin{solution}
Answer to the first problem.
\end{solution}

\begin{problem}
The second math calculation problem.
\end{problem}
\begin{solution}
Answer to the second problem.
\end{solution}
\end{code}

\begin{problem}
The first math calculation problem.
\end{problem}
\begin{solution}
Answer to the first problem.
\end{solution}

\begin{problem}
The second math calculation problem.
\end{problem}
\begin{solution}
Answer to the second problem.
\end{solution}

\subsection{Proof problems}

\begin{code}
\makepart{Work out math proofs.}{8 points for each.}

\begin{problem}
The first math proof problem.
\end{problem}
\begin{solution}
Answer to the first problem.
\end{solution}

\begin{problem}
The second math proof problem.
\end{problem}
\begin{solution}
Answer to the second problem.
\end{solution}
\end{code}

\begin{problem}
The first math proof problem.
\end{problem}
\begin{solution}
Answer to the first problem.
\end{solution}

\begin{problem}
The second math proof problem.
\end{problem}
\begin{solution}
Answer to the second problem.
\end{solution}

\subsection{Solution name}

If you want to change the name of \myenv{solution} environment,
you could redefine \mycmd{\solutionname} command.
The following example changes it from "Solution" to "Proof":

\begin{code}
\renewcommand{\solutionname}{Proof}
\end{code}

\subsection{Points command}

Inside \myenv{solution} environment, you could use \mycmd{\points} to give points for each step.
For example:

\begin{code}
\begin{solution}
$1+1=2$ \points{4}
$2+2=4$ \points{8}
\end{solution}
\end{code}

\begin{solution}
$1+1=2$ \points{4}
$2+2=4$ \points{8}
\end{solution}

You can also use \mycmd{\points} command inside displayed formulas or \myenv{align*} environment.
And the point text will be printed at the right edge of the line.

\subsection{Alignment commands}

With class option \myopt{many}, \mypkg{randexam} will load \mypkg{freealign} package.
and \mypkg{freealign} package provides several commands for aligning math formulas in different lines.

Here is the first example:

\begin{code}
We have $(a+b)^2 \? = (a+b)(a+b)$ \\
                 \+$= a^2+2ab+b^2$ \points{2}
\end{code}

\hrule
We have $(a+b)^2 \? = (a+b)(a+b)$ \\
                 \+$= a^2+2ab+b^2$ \points{2}
\hrule\vskip0.5em

The \mycmd{\?} command \underline{inside} the first formula saves current horizontal position,
and the \mycmd{\+} command \underline{before} the second formula jumps to previously saved position.

Here is another example:

\begin{code}
We have \? $(a+b)^2 = (a+b)(a+b)$ \\
        \< $= a^2+2ab+b^2$ \points{2}
\end{code}

\hrule
We have \? $(a+b)^2 = (a+b)(a+b)$ \\
        \< $= a^2+2ab+b^2$ \points{2}
\hrule\vskip0.5em

The \mycmd{\?} command \underline{before} the first formula saves current horizontal position,
and the \mycmd{\<} command \underline{before} the second formula jumps to the left
of previously saved position by the width of $=$.

Because \mypkg{freealign} package uses \mypkg{zref} package to save positions,
you need two compilations to get correct results.

\subsection{Other problems}

You can write other types of problems. For example:

\begin{code}
\makepart{Some problem type}{4 points for each.}

\begin{problem}
First problem text. \answer{Answer text.}
\end{problem}

\begin{problem}
Second problem text. \answer{Answer text.}
\end{problem}
\end{code}

\begin{problem}
First problem text. \answer{Answer text.}
\end{problem}

\begin{problem}
Second problem text. \answer{Answer text.}
\end{problem}

You must put answer text inside \mycmd{\answer} command,
so that \mypkg{randexam} could hide them in generating empty exam.

\subsection{Appendix data}

At the end of the exam, you could add appendix data with \mycmd{\makedata} command:

\begin{code}
\makedata{Some data may be used in the exam}
......
\end{code}

You must put appendix data after \mycmd{\makedata} command,
or the exam variants will be incorrect.

\subsection{Vertical space}

You could leave some vertical space after a \myenv{problem} or \myenv{solution} environment.
At this time \mypkg{randexam} class supports the following commands for adding vertical space:\par

\renewcommand{\arraystretch}{1.3}%
\begin{tabularx}{\linewidth}{l<{\qquad}X}
  \hline
  \texttt{\string\smallskip} & Add small vertical space\\
  \hline
  \texttt{\string\medskip}   & Add medium vertical space\\
  \hline
  \texttt{\string\bigskip}   & Add big vertical space\\
  \hline
  \texttt{\string\vfill}     & Fill vertical space available\\
  \hline
\end{tabularx}\par

Of course, you could use multiple commands in the above tables.

In the exam body, you could use \mycmd{\newpage} to make a page break,
but you should \underline{NOT} use other page breaking commands, such as \mycmd{\clearpage},
or the exam variants may be wrong.

\section{Options}

\subsection{Empty exam paper}

Assume \myfile{exam-a-answer.tex} is an exam paper with answers.
You can easily get an empty exam paper with answers removed,
by creating an \myfile{exam-a-empty.tex} file with the following lines:

\begin{code}
\PassOptionsToClass{noanswer}{randexam}
\input{exam-a-answer}
\end{code}

That is to say, when adding \myopt{noanswer} option to \mypkg{randexam} class,
The answers will be hidden in the compiled exam paper.

\subsection{Randomized variants}

Assume \myfile{exam-a-answer.tex} is an exam paper.
You can get an randomized variant with all problems in the same group shuffled,
by creating an \myfile{exam-b-answer.tex} file with the following lines:

\begin{code}
\PassOptionsToClass{random}{randexam}
\input{exam-a-answer}
\end{code}

That is to say, when adding \myopt{random} option to \mypkg{randexam} class,
The problems in the same group will be shuffled in the compiled exam paper.
Furthermore, four choices in an \myenv{abcd} environment will be shuffled too.

\subsection{Two column exam}

Assume \myfile{exam-a-empty.tex} is the TeX file of an exam paper of A4 size.
You could get an exam paper of A3 size,
by creating a new TeX file with the following lines:

\begin{code}
\PassOptionsToClass{a3paper}{randexam}
\input{exam-a-empty}
\end{code}

That is to say, when adding \myopt{a3paper} option to \mypkg{randexam} class,
The result paper will be a two column document in A3 size.

Assume \myfile{exam-a-empty.pdf} is the PDF file of an exam paper of A4 size.
You could get an exam paper of A3 size,
by creating a new TeX file with the following lines:
\begin{code}

\documentclass[a3input]{randexam}
\begin{document}
\includepdf[pages=-,nup=2x1]{exam-a-empty}
\end{document}
\end{code}

That is to say, you can make an exam of A3 size from an exam of A4 size,
even if you have only the PDF file.

\end{document}
