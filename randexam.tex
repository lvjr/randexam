% -*- coding: utf-8 -*-

\documentclass[12pt,plain,most]{randexam}
\geometry{b5paper,margin=2cm}

\newcommand*{\myversion}{2024B}
\newcommand*{\mydate}{\the\year-\mylpad\month-\mylpad\day}
\newcommand*{\mylpad}[1]{\ifnum#1<10 0\the#1\else\the#1\fi}

\cfoot{\small\thepage}

\setlength{\parindent}{0pt}
\setlength{\parskip}{7pt plus 1pt minus 1pt}

\usepackage{arevtext}
\usepackage{iftex}
\usepackage{ninecolors}
\usepackage{hyperref}
\hypersetup{
  colorlinks=true,
  urlcolor=blue3,
  linkcolor=blue3,
}

\renewcommand\familydefault{\sfdefault}

\renewcommand{\baselinestretch}{1}
\renewcommand{\arraystretch}{1.3}

\usepackage{tabularx}

\newcommand{\fillbox}[1]{\ulinefill{#1}\underline{#1}\ulinefill{#1}}

\usepackage{fancyvrb}

\DefineVerbatimEnvironment{code}{Verbatim}{%
  formatcom=\color{blue!50!red}%
}

\usepackage{codehigh}

\NewDocumentCommand\mypkg{m}{\textcolor{blue3}{\mbox{#1}}}
\NewDocumentCommand\myopt{m}{\textcolor{yellow3}{\mbox{#1}}}
\NewDocumentCommand\mycmd{m}{\textcolor{green3}{\ttfamily\fakeverb{#1}}}
\NewDocumentCommand\myenv{m}{\textcolor{green3}{\ttfamily#1}}
\NewDocumentCommand\myfile{m}{\textcolor{purple3}{\mbox{#1}}}
\NewDocumentCommand\mykey{m}{\textcolor{brown3}{\mbox{\fakeverb{#1}}}}
\NewDocumentCommand\myval{m}{\textcolor{azure3}{\mbox{\fakeverb{#1}}}}

\begin{document}

{%
  \renewcommand{\arraystretch}{2}%
  \noindent\Large
  \begin{tabularx}{\linewidth}{|X|}
    \hline
    Title: \fillbox{\color{blue3}The randexam class for LaTeX}\\
    Author: \fillbox{Jianrui Lyu (tolvjr@163.com)}\\
    Version: \fillbox{\myversion{} (\mydate)}\\
    \hline
  \end{tabularx}%
}

\tableofcontents

\section{Introduction}

Document class \mypkg{randexam} is an exam class for LaTeX.
With this class you could easily make an exam paper and its randomized variants.

The class mainly focuses on making math exam papers, but you could use it to make other exam papers.

The latest release of this package can be downloaded from here: \newline
\textcolor{blue}{\href{https://ctan.org/pkg/randexam}{\ttfamily https://ctan.org/pkg/randexam}}.

\section{Basic Structure}

\subsection{A randexam document}

The following is the basic structure of a \mypkg{randexam} document:
\begin{code}
\documentclass{randexam}
% document preamble
\begin{document}
% document body
\end{document}
\end{code}
In document preamble you could set some options for the exam.
In document body you write the contents of the exam.

\subsection{Document preamble}

In document preamble you normally set some class options with \mycmd{\SetExamOption} command.
For example, you could change the random seed with \myopt{seed} key. The seed is used only when
you add class option \myopt{random} (see Subsection \ref{opt:random}):
\begin{code}
\SetExamOption{
  seed = 19061116, % random seed
}
\end{code}

\subsection{Document body}

In document body you normally write an \mycmd{\examtitle},
multiple \mycmd{\examgroup}, and an optional \mycmd{\examdata} commands:

\begin{code}
\examtitle{name=Math Final Exam,date=2018-06-28,version=A}
......
\examgroup{Fill in the blanks.}{3 points for each.}
......
\examgroup{Select one answer.}{3 points for each.}
......
\examgroup{Work out math calculations.}{8 points for each.}
......
\examdata{Some data may be used in the exam}
......
\end{code}

\subsection{Exam title}

In calling \mycmd{\examtitle} command,
you need to provide some basic informations of the exam:

\begin{code}
\examtitle{name=Math Final Exam,date=2018-06-28,version=A}
\end{code}

If the \mykey{date} key is missing, it has the default value \myval{\today}.
When class option \myopt{random} is passed, the value of \mykey{version} key
will be modified from \myval{A} to \myval{B}.

\subsection{Exam groups}

The questions in an exam could be separated into several groups:

\begin{code}
\examgroup{Fill in the blanks.}{3 points for each.}
......
\examgroup{Select one answer.}{3 points for each.}
......
\examgroup{Work out math calculations.}{8 points for each.}
......
\end{code}

\subsection{Exam data}

At the end of the exam, you could add some appendix data with \mycmd{\examdata} command:

\begin{code}
\examdata{Some data may be used in the exam}
......
\end{code}

You must put appendix data after \mycmd{\examdata} command,
or the exam variants will be incorrect.

\section{Types of questions}

\subsection{True-or-false questions}

\begin{code}
\examgroup{True-or-false questions}{3 points for each.}

\begin{question}
The first true-or-false question. \tickout{T}
\end{question}

\begin{question}
The second true-or-false question. \tickout{F}
\end{question}
\end{code}

\begin{question}
The first true-or-false question. \tickout{T}
\end{question}

\begin{question}
The second true-or-false question. \tickout{F}
\end{question}

With \mycmd{\tickout{T}} and \mycmd{\tickout{F}}, you get \textcolor{blue}{\textsf{T}}
and \textcolor{blue}{\textsf{F}}; with \mycmd{\tickout{t}} and \mycmd{\tickout{f}},
you get \textcolor{blue}{$\checkmark$} and \textcolor{blue}{\large$\times$}.

You must put answers inside \mycmd{\tickout} command,
so as to make \mypkg{randexam} hide them in generating blank exam papers.

\subsection{Fill-in-the-blank questions}

\begin{code}
\examgroup{Fill in the blanks.}{3 points for each.}

\begin{question}
The first fill-in-the-blank question \fillout{answer}.
\end{question}

\begin{question}
The second fill-in-the-blank question \fillout{answer}.
\end{question}
\end{code}

\begin{question}
The first fill-in-the-blank question \fillout{answer}.
\end{question}

\begin{question}
The second fill-in-the-blank question \fillout{answer}.
\end{question}

With \mycmd{\fillout} command, the underline will fill the whole line;
with \mycmd{\fillin} command, the underline will be minimal.

You must put answers inside \mycmd{\fillout} or \mycmd{\fillin} command,
so as to make \mypkg{randexam} hide them in generating blank exam papers.

\subsection{Multiple-choice questions}

\begin{code}
\examgroup{Select one answer.}{3 points for each.}

\begin{question}
The first multiple-choice questions \pickout{A}.
\begin{abcd}
  \item First
  \item Second
  \item Third
  \item Fourth
\end{abcd}
\end{question}

\begin{question}
The second multiple-choice questions \pickout{C}.
\begin{abcd}
  \item First choice
  \item Second choice
  \item Third choice
  \item Fourth choice
\end{abcd}
\end{question}
\end{code}

\begin{question}
The first multiple-choice questions \pickout{A}.
\begin{abcd}
  \item First
  \item Second
  \item Third
  \item Fourth
\end{abcd}
\end{question}

\begin{question}
The second multiple-choice questions \pickout{C}.
\begin{abcd}
  \item First choice
  \item Second choice
  \item Third choice
  \item Fourth choice
\end{abcd}
\end{question}

With \mycmd{\pickout} command, the answer will be printed on the right edge of the line;
with \mycmd{\pickin} command, the answer will be printed on current position.

You must put answers inside \mycmd{\pickout} or \mycmd{\pickin} command,
so as to make \mypkg{randexam} hide them in generating blank exam papers.

The four choices of multiple-choice questions could be typeset with \myenv{abcd} environment.
And \myenv{abcd} environment will put them in one, two, or four rows
according to the lengths of the choices.

\subsection{Subjective questions}

For subjective questions, you could put answers inside \myenv{solution} environment.

\begin{code}
\examgroup{Work out math calculations.}{8 points for each.}

\begin{question}
The first math calculation question.
\end{question}

\begin{solution}
Answer to the first question.
\end{solution}

\begin{question}
The second math calculation question.
\end{question}

\begin{solution}
Answer to the second question.
\end{solution}
\end{code}

\begin{question}
The first math calculation question.
\end{question}

\begin{solution}
Answer to the first question.
\end{solution}

\begin{question}
The second math calculation question.
\end{question}

\begin{solution}
Answer to the second question.
\end{solution}

\subsection{Other questions}

You can write other types of questions. For example:

\begin{code}
\examgroup{Some question type}{4 points for each.}

\begin{question}
First question text. \answer{Answer text.}
\end{question}

\begin{question}
Second question text. \answer{Answer text.}
\end{question}
\end{code}

\begin{question}
First question text. \answer{Answer text.}
\end{question}

\begin{question}
Second question text. \answer{Answer text.}
\end{question}

You must put answer text inside \mycmd{\answer} command,
so as to make \mypkg{randexam} hide them in generating blank exam papers.

\section{Other components}

\subsection{Grade tables}

Normally you need a grade table after exam title.
You can get it by using \mycmd{\gradetable} command.

\begin{code}
\gradetable
\end{code}

\gradetable

\subsection{Answer tables}

Before true-or-false, fill-in-the-blank, or multiple-choice questions,
you may use \mycmd{\answertable} to generate an blank answer table:

\begin{code}
\answertable[3em]{6}{3}
\end{code}

\answertable[3em]{6}{3}

The meanings of the three arguments of \verb!\answertable! commands are as follows:
\begin{itemize}
  \item The first one means the strut height of the answer rows; its default value is \verb!1em!.
  \item The second one means the total number of questions in this exam group.
  \item The third one means the number of questions in each row.
\end{itemize}

\subsection{Vertical space}

You could leave some vertical space after a \myenv{question} or \myenv{solution} environment.
At this time \mypkg{randexam} class supports the following commands for adding vertical space:\par

\renewcommand{\arraystretch}{1.3}%
\begin{tabularx}{\linewidth}{l<{\qquad}X}
  \hline
  \texttt{\string\smallskip} & Add small vertical space\\
  \hline
  \texttt{\string\medskip}   & Add medium vertical space\\
  \hline
  \texttt{\string\bigskip}   & Add big vertical space\\
  \hline
  \texttt{\string\vfill}     & Fill vertical space available\\
  \hline
\end{tabularx}\par

Of course, you could use multiple commands in the above tables.

In the exam body, you could use \mycmd{\newpage} to make a page break,
but you should \underline{NOT} use other page breaking commands, such as \mycmd{\clearpage},
or the exam variants may be wrong.

\section{Writing questions}

\subsection{points key}

The \myenv{question} environment accepts a \mykey{points} key:

\begin{code}
\begin{question}[points=4]
This is a question.
\end{question}
\end{code}

\begin{question}[points=4]
This is a question.
\end{question}

\section{Writing solutions}

\subsection{Solution name}

If you want to change the name of \myenv{solution} environment,
you could redefine \mycmd{\solutionname} command.
The following example changes it from "Solution" to "Proof":

\begin{code}
\renewcommand{\solutionname}{Proof}
\end{code}

\subsection{Points command}

Inside \myenv{solution} environment, you could use \mycmd{\points} to give points for each step.
For example:

\begin{code}
\begin{solution}
$1+1=2$ \points{4}
$2+2=4$ \points{8}
\end{solution}
\end{code}

\begin{solution}
$1+1=2$ \points{4}
$2+2=4$ \points{8}
\end{solution}

You can also use \mycmd{\points} command inside displayed formulas or \myenv{align*} environment.
And the point text will be printed at the right edge of the line.

\subsection{Alignment commands}

With class option \myopt{many}, \mypkg{randexam} will load \mypkg{freealign} package.
and \mypkg{freealign} package provides several commands for aligning math formulas in different lines.

Here is the first example:

\begin{code}
We have $(a+b)^2 \? = (a+b)(a+b)$ \\
                 \+$= a^2+2ab+b^2$ \points{2}
\end{code}

\hrule
We have $(a+b)^2 \? = (a+b)(a+b)$ \\
                 \+$= a^2+2ab+b^2$ \points{2}
\hrule\vskip0.5em

The \mycmd{\?} command \underline{inside} the first formula saves current horizontal position,
and the \mycmd{\+} command \underline{before} the second formula jumps to previously saved position.

Here is another example:

\begin{code}
We have \? $(a+b)^2 = (a+b)(a+b)$ \\
        \< $= a^2+2ab+b^2$ \points{2}
\end{code}

\hrule
We have \? $(a+b)^2 = (a+b)(a+b)$ \\
        \< $= a^2+2ab+b^2$ \points{2}
\hrule\vskip0.5em

The \mycmd{\?} command \underline{before} the first formula saves current horizontal position,
and the \mycmd{\<} command \underline{before} the second formula jumps to the left
of previously saved position by the width of $=$.

Because \mypkg{freealign} package uses \mypkg{zref} package to save positions,
you need two compilations to get correct results.

\section{Class options}

\subsection{Blank exam papers}

Assume \myfile{exam-a-answer.tex} is an exam paper with answers.
You can easily get an blank exam paper with answers removed,
by creating an \myfile{exam-a-blank.tex} file with the following lines:

\begin{code}
\PassOptionsToClass{noanswer}{randexam}
\input{exam-a-answer}
\end{code}

That is to say, when adding \myopt{noanswer} option to \mypkg{randexam} class,
The answers will be hidden in the compiled exam paper.

\subsection{Randomized variants}\label{opt:random}

Assume \myfile{exam-a-answer.tex} is an exam paper.
You can get an randomized variant with all questions in the same group shuffled,
by creating an \myfile{exam-b-answer.tex} file with the following lines:

\begin{code}
\PassOptionsToClass{random}{randexam}
\input{exam-a-answer}
\end{code}

That is to say, when adding \myopt{random} option to \mypkg{randexam} class,
The questions in the same group will be shuffled in the compiled exam paper.
Furthermore, four choices in an \myenv{abcd} environment will be shuffled too.

\subsection{Two column exam papers}

Assume \myfile{exam-a-blank.tex} is the TeX file of an exam paper of A4 size.
You could get an exam paper of A3 size,
by creating a new TeX file with the following lines:

\begin{code}
\PassOptionsToClass{a3paper}{randexam}
\input{exam-a-blank}
\end{code}

That is to say, when adding \myopt{a3paper} option to \mypkg{randexam} class,
The result paper will be a two column document in A3 size.

Assume \myfile{exam-a-blank.pdf} is the PDF file of an exam paper of A4 size.
You could get an exam paper of A3 size,
by creating a new TeX file with the following lines:
\begin{code}

\documentclass[a3input]{randexam}
\begin{document}
\includepdf[pages=-,nup=2x1]{exam-a-blank}
\end{document}
\end{code}

That is to say, you can make an exam of A3 size from an exam of A4 size,
even if you have only the PDF file.

\end{document}
